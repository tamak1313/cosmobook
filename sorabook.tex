% Options for packages loaded elsewhere
\PassOptionsToPackage{unicode}{hyperref}
\PassOptionsToPackage{hyphens}{url}
%
\documentclass[
]{book}
\usepackage{lmodern}
\usepackage{amssymb,amsmath}
\usepackage{ifxetex,ifluatex}
\ifnum 0\ifxetex 1\fi\ifluatex 1\fi=0 % if pdftex
  \usepackage[T1]{fontenc}
  \usepackage[utf8]{inputenc}
  \usepackage{textcomp} % provide euro and other symbols
\else % if luatex or xetex
  \usepackage{unicode-math}
  \defaultfontfeatures{Scale=MatchLowercase}
  \defaultfontfeatures[\rmfamily]{Ligatures=TeX,Scale=1}
\fi
% Use upquote if available, for straight quotes in verbatim environments
\IfFileExists{upquote.sty}{\usepackage{upquote}}{}
\IfFileExists{microtype.sty}{% use microtype if available
  \usepackage[]{microtype}
  \UseMicrotypeSet[protrusion]{basicmath} % disable protrusion for tt fonts
}{}
\makeatletter
\@ifundefined{KOMAClassName}{% if non-KOMA class
  \IfFileExists{parskip.sty}{%
    \usepackage{parskip}
  }{% else
    \setlength{\parindent}{0pt}
    \setlength{\parskip}{6pt plus 2pt minus 1pt}}
}{% if KOMA class
  \KOMAoptions{parskip=half}}
\makeatother
\usepackage{xcolor}
\IfFileExists{xurl.sty}{\usepackage{xurl}}{} % add URL line breaks if available
\IfFileExists{bookmark.sty}{\usepackage{bookmark}}{\usepackage{hyperref}}
\hypersetup{
  pdftitle={sorabook},
  pdfauthor={Tamaki.M},
  hidelinks,
  pdfcreator={LaTeX via pandoc}}
\urlstyle{same} % disable monospaced font for URLs
\usepackage{longtable,booktabs}
% Correct order of tables after \paragraph or \subparagraph
\usepackage{etoolbox}
\makeatletter
\patchcmd\longtable{\par}{\if@noskipsec\mbox{}\fi\par}{}{}
\makeatother
% Allow footnotes in longtable head/foot
\IfFileExists{footnotehyper.sty}{\usepackage{footnotehyper}}{\usepackage{footnote}}
\makesavenoteenv{longtable}
\usepackage{graphicx,grffile}
\makeatletter
\def\maxwidth{\ifdim\Gin@nat@width>\linewidth\linewidth\else\Gin@nat@width\fi}
\def\maxheight{\ifdim\Gin@nat@height>\textheight\textheight\else\Gin@nat@height\fi}
\makeatother
% Scale images if necessary, so that they will not overflow the page
% margins by default, and it is still possible to overwrite the defaults
% using explicit options in \includegraphics[width, height, ...]{}
\setkeys{Gin}{width=\maxwidth,height=\maxheight,keepaspectratio}
% Set default figure placement to htbp
\makeatletter
\def\fps@figure{htbp}
\makeatother
\setlength{\emergencystretch}{3em} % prevent overfull lines
\providecommand{\tightlist}{%
  \setlength{\itemsep}{0pt}\setlength{\parskip}{0pt}}
\setcounter{secnumdepth}{5}
\usepackage{booktabs}
\usepackage[]{natbib}
\bibliographystyle{apalike}

\title{sorabook}
\author{Tamaki.M}
\date{2020-10-12}

\begin{document}
\maketitle

{
\setcounter{tocdepth}{1}
\tableofcontents
}
\hypertarget{ux521dux3081ux306b}{%
\chapter*{初めに}\label{ux521dux3081ux306b}}
\addcontentsline{toc}{chapter}{初めに}

近年、衛星データは注目を集めています。

\hypertarget{ux60f3ux5b9aux8aadux8005}{%
\section{想定読者}\label{ux60f3ux5b9aux8aadux8005}}

\begin{itemize}
\tightlist
\item
  衛星データとは何か知りたい人
\item
  衛星データを使ってみたい人
\end{itemize}

\hypertarget{ux4f5cux696dux74b0ux5883}{%
\section{作業環境}\label{ux4f5cux696dux74b0ux5883}}

2020年10月現在\\
Windows 10 64bit\\
QGIS 3.14.16

\hypertarget{ux885bux661fux30c7ux30fcux30bfux306eux6d3bux7528ux4e8bux4f8b}{%
\chapter{衛星データの活用事例}\label{ux885bux661fux30c7ux30fcux30bfux306eux6d3bux7528ux4e8bux4f8b}}

\hypertarget{ux885bux661fux30c7ux30fcux30bfux306eux7a2eux985eux3092ux77e5ux308b}{%
\chapter{衛星データの種類を知る}\label{ux885bux661fux30c7ux30fcux30bfux306eux7a2eux985eux3092ux77e5ux308b}}

\hypertarget{ux885bux661fux30c7ux30fcux30bfux306eux53d6ux5f97ux65b9ux6cd5usgs}{%
\chapter{衛星データの取得方法(USGS)}\label{ux885bux661fux30c7ux30fcux30bfux306eux53d6ux5f97ux65b9ux6cd5usgs}}

\hypertarget{usgs-earthexplorerux3092ux30c0ux30a6ux30f3ux30edux30fcux30c9ux767bux9332ux30edux30b0ux30a4ux30f3ux3059ux308b}{%
\section{USGS EarthExplorerをダウンロード、登録/ログインする}\label{usgs-earthexplorerux3092ux30c0ux30a6ux30f3ux30edux30fcux30c9ux767bux9332ux30edux30b0ux30a4ux30f3ux3059ux308b}}

\hypertarget{ux30c7ux30fcux30bfux3092ux53d6ux5f97ux3057ux305fux3044ux5730ux57dfux3092ux8a2dux5b9aux3059ux308b}{%
\section{データを取得したい地域を設定する}\label{ux30c7ux30fcux30bfux3092ux53d6ux5f97ux3057ux305fux3044ux5730ux57dfux3092ux8a2dux5b9aux3059ux308b}}

\begin{itemize}
\item
  【Geocoder】のタブで【World Features】を選択する(米国内のデータを取得したい場合は、【US Features(デフォルト)】に設定)
\item
  【Country】から【BANGLADESH】を選択
\item
  【Show】を押下
\item
  表示された表から、取得したい地域に近い/取得したい地域を選択する
\end{itemize}

下図のように、右エリアにポイントが表示される

\hypertarget{ux30c7ux30fcux30bfux53d6ux5f97ux9818ux57dfux3092ux30d0ux30f3ux30b0ux30e9ux30c7ux30b7ux30e5ux5168ux4f53ux306bux62e1ux5f35ux3059ux308b}{%
\section{データ取得領域をバングラデシュ全体に拡張する}\label{ux30c7ux30fcux30bfux53d6ux5f97ux9818ux57dfux3092ux30d0ux30f3ux30b0ux30e9ux30c7ux30b7ux30e5ux5168ux4f53ux306bux62e1ux5f35ux3059ux308b}}

【Polygon】\\
+ データを取得したいエリアをポイントが取り囲むよう、各地点をクリックしていく
+ 選択したエリアが赤く表示される

【Circle】\\
+ データを取得したい地域を含むように、右地図上の2点を選択する
+ 選択した2点で描かれる円の範囲が赤く表示される

\hypertarget{ux30c7ux30fcux30bfux3092ux53d6ux5f97ux3057ux305fux3044ux65e5ux6642ux3092ux8a2dux5b9aux3059ux308b}{%
\section{データを取得したい日時を設定する}\label{ux30c7ux30fcux30bfux3092ux53d6ux5f97ux3057ux305fux3044ux65e5ux6642ux3092ux8a2dux5b9aux3059ux308b}}

\begin{itemize}
\tightlist
\item
  【Date Range】から選択
  (カレンダーから選択が可能)
\end{itemize}

\hypertarget{ux96f2ux306eux91cfux3092ux8abfux6574ux3059ux308b}{%
\section{雲の量を調整する}\label{ux96f2ux306eux91cfux3092ux8abfux6574ux3059ux308b}}

\begin{itemize}
\tightlist
\item
  【Cloud Cover】でバーを動かすことにより、設定が可能
  *ただし衛星データによっては雲量調整を行っていないものがあり、その場合は適用されない
\end{itemize}

\hypertarget{data-setsux3092ux62bcux4e0b}{%
\section{Data Setsを押下}\label{data-setsux3092ux62bcux4e0b}}

\hypertarget{ux5fc5ux8981ux306aux30c7ux30fcux30bfux30bbux30c3ux30c8ux3092ux9078ux629eux3059ux308b}{%
\section{必要なデータセットを選択する}\label{ux5fc5ux8981ux306aux30c7ux30fcux30bfux30bbux30c3ux30c8ux3092ux9078ux629eux3059ux308b}}

\begin{itemize}
\tightlist
\item
  データセットの中から必要なデータセットを選択する\\
  *データセット名の左のインフォメーションマークを押下すると説明が出る\\
  *データセット名の左の地図マークを押下すると、そのデータのカバー範囲が右地図上に表示される
\end{itemize}

\hypertarget{resultsux3092ux62bcux4e0bux3059ux308b}{%
\section{Resultsを押下する}\label{resultsux3092ux62bcux4e0bux3059ux308b}}

\hypertarget{ux30c7ux30fcux30bfux306eux30c0ux30a6ux30f3ux30edux30fcux30c9}{%
\section{データのダウンロード}\label{ux30c7ux30fcux30bfux306eux30c0ux30a6ux30f3ux30edux30fcux30c9}}

+【Show Brouwse Overlay】を押下すると、各データを右地図上に表示することができる\\
*複数のデータにわたる場合には、Acquisition Dateを確認する(データの詳細は【Show Mwtadata and Browse】から確認可能)
+ 必要なデータセット以外は、【Exclude Scene from Results】を押下
+ 【Click here to export your results】の【Export data】を押下し、【Export Name】、【Export Format】を設定する\\
ダウンロード完了

\hypertarget{applications}{%
\chapter{Applications}\label{applications}}

Some \emph{significant} applications are demonstrated in this chapter.

\hypertarget{example-one}{%
\section{Example one}\label{example-one}}

\hypertarget{example-two}{%
\section{Example two}\label{example-two}}

\hypertarget{final-words}{%
\chapter{Final Words}\label{final-words}}

We have finished a nice book.

  \bibliography{book.bib,packages.bib}

\end{document}
